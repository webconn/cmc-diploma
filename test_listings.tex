\section{Исходные данные и результаты экспериментов}
\label{sec:test_listings}

\subsection{Генератор трасс}
\label{subsec:generator}

Для проверки общей работоспособности решения без доступа к реальным 
или смоделированным ВС полезно провести апробацию средства на данных, 
полученных с помощью генератора трасс обменов.

Такой генератор был разработан ранее для демонстрационных целей. Для 
передачи данных агенту Анализатора MIL STD-1553B используется специальное 
виртуальное 
устройство-петля. С помощью генератора можно получить последовательности 
обменов, подходящие для проверки возможностей анализатора ограничений.

Во время работы генератор непрерывно передаёт набор сообщений определённых 
типов и содержания, эмулируя работу нескольких устройств на канале.

Перечень генерируемых параметров приведён в таблице~\ref{tab:params}. Для 
каждого параметра приводится функция значения, а также описание 
обмена, передающего значение параметра (адреса и подадреса отправителя и 
получателя, формат сообщения, размер сообщения, номер первого слова, номер 
первого бита и длина битового поля). Параметры по умолчанию целочисленные.

Значение $i$ в определении функции - значение целочисленного счётчика, 
увеличиваемое на 1 каждый промежуток времени, заданный в исходном коде 
генератора (по умолчанию - 300 мс, что даёт частоту обновления значения 
параметра около 3.3 Гц).

Значение $crc(1:6)$ - значение контрольной суммы сообщения, получаемого 
абонентом с адресом 1 и подадресом 6 от контроллера канала. Контрольная сумма 
считается от слов 2-4 и располагается в первом слове сообщения.

Легенда таблицы:

\begin{itemize}
 \item $A_{src}$ - адрес источника сообщения (0-31, либо КК - контроллер 
канала);
 \item $SA_{src}$ - подадрес источника сообщения (0-30, либо КК - контроллер 
канала);
 \item $A_{dst}$ - адрес получателя сообщения (0-31, либо КК - контроллер 
канала);
 \item $SA_{dst}$ - подадрес получателя сообщения (0-30, либо КК - контроллер 
канала);
 \item $Sz_{msg}$ - размер сообщения (количество слов);
 \item $St_{word}$ - номер слова, в котором начинается битовое поле параметра;
 \item $St_{bit}$ - номер первого бита битового поля параметра в этом слове;
 \item $Sz_{bf}$ - количество бит в битовом поле параметра;
 \item $Sc$ - цена младшего бита битового поля.
\end{itemize}

\begin{table}[H]
\centering
\begin{tabular}{|l|l|*{9}{c|}}
\hline
 № & Функция & $A_{src}$ & $SA_{src}$ & $A_{dst}$ & $SA_{dst}$ & $Sz_{msg}$ & 
$St_{word}$ & $St_{bit}$ & $Sz_{bf}$ & $Sc$ \\
\hline
1 & $10^6 * sin(i / 10)$ & КК & КК & 2 & 1 & 4 & 0 & 0 & 32 & 1.0 \\
2 & $i$; 0x42 & КК & КК & 2 & 1 & 4 & 2 & 0 & 16 & 1.0 \\
3 & $10^6 * sin(i / 10)$ & КК & КК & 2 & 3 & 4 & 0 & 0 & 32 & 1.0 \\
4 & $i$; 0x42 & КК & КК & 2 & 3 & 4 & 2 & 0 & 16 & 1.0 \\
5 & $10^6 * sin(i / 10)$ & КК & КК & 2 & 4 & 4 & 0 & 0 & 32 & 1.0 \\
6 & $i$; 0x42 & КК & КК & 2 & 4 & 4 & 2 & 0 & 16 & 1.0 \\
7 & $crc(0x01:0x06)$ & 1 & 6 & КК & КК & 4 & 0 & 0 & 16 & 1.0 \\
\hline 
\end{tabular}
\caption{Перечень параметров, предоставляемых генератором}
\label{tab:params}
\end{table}

В значения некоторых параметров генератором вносятся ошибки. Ошибки 
для таких параметров описаны в таблице~\ref{tab:param_errors}. Частота 
возникновения здесь - количество циклов генератора, определяемых частотой 
увеличения внутреннего целочисленного счётчика $i$. $rnd()$ - значение, 
определяемое с помощью генератора псевдослучайных чисел (функции 
\textit{random()} стандартной библиотеки языка Си).

\begin{table}[H]
 \centering
 \begin{tabular}{|l|c|c|}
  \hline
  № & Ошибочное значение & Частота возникновения \\
  \hline
  5 & побитовое НЕ верного значения & $rnd()$ \\
  6 & 0xDEAD & 10 \\
  7 & $crc(1:6) + 1$ & $rnd()$ \\
  \hline
 \end{tabular}
 \caption{Ошибочные значения параметров}
 \label{tab:param_errors}
\end{table}


Ниже приводятся примеры описаний протоколов, графики значений параметров, 
описанных в этих протоколах и результаты работы анализатора для этих 
описаний протоколов. Проверка работы анализатора проводилась в режиме 
регистрации обменов в реальном времени.

\subsection{Исходные данные и результаты тестирования}

\subsubsection{Пример 1}

В примере 1 наблюдается изменение младших 8 бит значений параметров 4 
и 6 из таблицы~\ref{tab:params} (назовём полученные параметры 4а и 6а 
соответственно). Значение параметра 4а 
постоянно и равно 42. Если для параметра 6а не выполнено условие генерации 
ошибки, его значение равно значению параметра 4а, иначе - константе 173 (0xAD). 

Параметры связываются в группу с идентификатором \textit{int42} со 
значениями погрешности по умолчанию (5\%). Поскольку параметр 6а принимает 
ошибочные значения со значительным отклонением (больше 5\%), в отчёте 
анализатора должны появиться сообщения о расхождении значений параметров в 
группе \textit{int42}.

Также в примере 1 проверяются свойства сообщений. Для проверки частоты 
появления обмена используется сообщение, содержащее параметр 4. Для проверки 
флагов ошибок выбрано сообщение, для которого генератор иногда не публикует 
ответного слова. Для проверки значения контрольной суммы используется 
сообщение, содержащее параметр 7. Последовательность обменов проверяется для 
сообщений, содержащих параметры 4 и 6. Для демонстрации нарушения порядка 
следования этих сообщений в регистрации сделана пауза, при этом одно из 
сообщений повторяется с пропуском второго.

\lstinputlisting[caption=Пример описания протокола 
1]{tests/param_bind/protocol.xml}

\begin{figure}[H]
 \centering
 \includegraphics[scale=0.6]{tests/param_bind/graph}
 \caption{График изменения значений параметров (красный - 4а, зелёный - 6а)}
 \label{fig:param_bind_graph}
\end{figure}

\begin{figure}[H]
 \centering
 \includegraphics[scale=0.4]{tests/param_bind/report}
 \caption{Результат работы анализатора для протокола 1}
 \label{fig:param_bind_report}
\end{figure}

\subsubsection{Пример 2}

В примере 2 наблюдается изменение значения параметра 6 из 
таблицы~\ref{tab:params}. 

Для проверки факта равенства младшего байта параметра ошибочному значению 173 
(0xAD) используется ограничение error\_value. Старший байт параметра 
рассматривается как автоматически увеличивающееся значение-счётчик (имеющее 
выбросы). В регистрации трассы были сделаны паузы для того, чтобы 
протестировать проверку опоздания момента увеличения значения счётчика.

\lstinputlisting[caption=Пример описания протокола 
2]{tests/param_error_value/protocol.xml}

\begin{figure}[H]
 \centering
 \includegraphics[scale=0.6]{tests/param_error_value/graph}
 \caption{График изменения значений параметра 6}
 \label{fig:param_error_value_graph}
\end{figure}

\begin{figure}[H]
 \centering
 \includegraphics[scale=0.4]{tests/param_error_value/report}
 \caption{Результат работы анализатора}
 \label{fig:param_error_value_report}
\end{figure}

\subsubsection{Пример 3}

В примере 3 наблюдается изменение значения параметра 6 из 
таблицы~\ref{tab:params}. Для обнаружения ``выбросов'' (значительных (>5\%)
отклонений значения параметра от исходной функции) используется ограничение 
smooth с максимальным значением производной по модулю, равным 1000 (значение 
подобрано эмпирически). Также в этом примере провеяется корректность 
определения выхода значения параметра за заданные пороги (для этого же 
параметра).

\lstinputlisting[caption=Пример описания протокола 
3]{tests/param_smooth/protocol.xml}

\begin{figure}[H]
 \centering
 \includegraphics[scale=0.6]{tests/param_smooth/graph}
 \caption{График изменения значений параметра 6}
 \label{fig:param_smooth_graph}
\end{figure}

\begin{figure}[H]
 \centering
 \includegraphics[scale=0.4]{tests/param_smooth/report}
 \caption{Результат работы анализатора}
 \label{fig:param_smooth_report}
\end{figure}