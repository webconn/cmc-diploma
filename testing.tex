\section{Результаты апробации решения}

%TODO: описать, откуда взяты данные

\subsection{Порядок проведения апробации}

В силу того, что реальные бортовые РВС являются закрытыми (иногда секретными) 
разработками, проверка полученного решения на реальных системах не 
представляется возможным. Поэтому для проведения апробации были использованы 
тестовые данные, полученные с помощью генератора трасс обменов.

Генератор трасс обменов эмулирует работу нескольких абонентов и контроллера 
канала на специальном виртуальном адаптере-петле. В ходе работы генератора 
устройства ``обмениваются'' значениями некоторых параметров (частичный перечень 
которых приведён в приложении~\ref{sec:test_listings}).

Изначально генератор использовался для демонстрации работы анализатора при 
физическом отсутствии подключения к шине. В данной работе генератор был 
модифицирован для того, чтобы провести проверку корректности обработки всех 
требований, сформулированных в работе.

В ходе апробации полученного решения проводились следующие проверки:

\begin{itemize}
 \item проверка корректности работы анализатора - для заданной трассы и 
описания требований проверялось обнаружение несоответствий и отсутствие ложных 
срабатываний анализатора;
 \item проверка быстродействия анализатора для определения возможности работы в 
режиме регистрации обмена в реальном времени.
\end{itemize}

Подробное описание характеристик генератора приведено в 
приложении~\ref{sec:test_listings}.

\subsection{Проверка корректности работы анализатора}

Апробация возможностей анализатора проводилась на наборах требований, 
составленных для трасс обменов от генератора. 
Подробные описания примеров тестовых протоколов приведены в 
приложении~\ref{sec:test_listings}.

\subsubsection{Пример 1}

В примере 1 проводится проверка следующих классов требований:

\begin{itemize}
 \item частота появления сообщения в канале;
 \item наличие в сообщении флагов ошибок, специфических для MIL STD-1553B;
 \item значение контрольной суммы в полезной нагрузке сообщения для MIL 
STD-1553B;
 \item последовательность появления сообщений из группы;
 \item минимальная частота обновления значения параметра;
 \item равенство значения параметра константе;
 \item проверка равенства значений группы связанных параметров.
\end{itemize}

\subsubsection{Пример 2}

В примере 2 проверяются следующие классы требований:

\begin{itemize}
 \item ошибочное значение параметра;
 \item автоматическое увеличение значения параметра (автоинкремент).
\end{itemize}

\subsubsection{Пример 3}

В примере 3 проверяются следующие классы требований:

\begin{itemize}
 \item гладкость значения параметра во времени;
 \item проверка выхода значения параметра за пороговые значения (в протоколе 
приведено несколько пороговых значений с разными направлениями и уровнями).
\end{itemize}

\subsubsection{Результат апробации}

В результате проверки были успешно обнаружены нарушения всех классов 
требований, описанных в данной работе, при этом ложные срабатывания анализатора 
не наблюдались.

\subsection{Проверка быстродействия анализатора}

Для проведения проверки быстродействия анализатора был модифицирован генератор 
обменов, использованный для предыдущих проверок. Целью модификации 
было получение большего числа типов сообщений.

Для проведения проверки записана трасса обменов с помощью  
модифицированного генератора длительностью 20 минут, содержащая $N_{exch} 
= 108873$ обмена. 

Файл протокола содержит описания сообщений, получаемых от 
модифицированного генератора (всего до 250 типов сообщений). 
Максимальное количество типов сообщений выбрано в соответствии с таковым в 
изученных трассах обменов, полученных с реальных РВС. Для каждого из типов 
сообщений описываются ограничения; количество ограничений для каждого из 
проведённых тестов приводится в таблице~\ref{tab:prof}.

Для типов сообщений, подразумевающих передачу полезной нагрузки, описаны также 
параметры и ограничения для них. Количество параметров в протоколе 
и количество ограничений для каждого из параметров приведены в 
таблице~\ref{tab:prof}.

Проверка проводилась для разного количества описаний сообщений и 
параметров, а также разного количества ограничений на сообщения и параметры.

Проект собирался с выставленными флагами оптимизации (перед сборкой была 
создана глобальная переменная окружения CL\_OPTIMIZE=1, используемая системой 
сборки cvslvk [\ref{cvslvk}]).

Конфигурация тестовой ЭВМ:

\begin{itemize}
 \item Процессор: Intel Core i5-3320M, 2.6 ГГц;
 \item ОЗУ: 1024 МБ;
 \item ОС: Debian Linux 7 (wheezy), 32-разрядная;
 \item SSD. 
\end{itemize}

Тестирование решения проводилось на виртуальной ЭВМ в среде Oracle VM 
VirtualBox [\ref{virtualbox}].

Для каждого из описаний требований проводилась операция загрузки трассы. 
Операция проводилась 5 раз для каждого описания требований, после чего 
подсчитывалось среднее значение времени загрузки трассы и среднее значение 
времени, потраченного в контексте анализатора. \textit{Время в контексте 
анализатора} - время, потраченное в методах-обработчиках:
\begin{itemize}
 \item AnalyzerCore::onParamValueUpdated(),
 \item AnalyzerCore::onAddExchange(). 
\end{itemize}

Время загрузки трассы оценивается временем выполнения метода 

\begin{itemize}
 \item ExchangeContainer::addSubscriber().
\end{itemize}

Для 5 запусков решения среднеквадратичное отклонение результатов 
измерения оценки времени выполнения для обоих случаев не превысило 2.5\%.

Результаты проверки приведены в таблице~\ref{tab:prof}.

Легенда таблицы:

\begin{itemize}
 \item $N_{msgs}$ - общее число сообщений, описанных в файле протокола;
 \item $N_{params}$ - общее число параметров, описанных в файле протокола;
 \item $M_{msg}$ - количество ограничений на каждое из описанных сообщений;
 \item $M_{param}$ - количество ограничений на каждый из описанных параметров;
 \item $t_{total}$ - среднее значение оценки времени загрузки трассы с данным 
описанием протокола (в секундах);
 \item $t_{a}$ - среднее значение оценки времени, потраченного в контексте 
анализатора при загрузке трассы с данным описанием протокола (в секундах);
 \item $\%$ - доля времени, потраченного в контексте анализатора, по 
отношению к общему времени загрузки трассы (в процентах, для средних значений 
оценки времени).
\end{itemize}


\begin{table}[H]
 \centering
 \begin{tabular}{|c|*{7}{c|}}
  \hline
  № теста & $N_{msgs}$ & $N_{params}$ & $M_{msg}$ & $M_{param}$ & $t_{total}$, 
с & $t_{a}$, с & \% \\
  \hline
  1 & 128 & 128 & 5   & 5   & 3.610892 & 0.572862 & 15.86 \\
  2 & 128 & 240 & 5   & 5   & 5.470608 & 0.888364 & 16.23 \\
  3 & 240 & 128 & 5   & 5   & 3.517034 & 0.553243 & 15.73 \\
  4 & 240 & 240 & 5   & 5   & 5.459458 & 0.889838 & 16.29 \\
  5 & 128 & 128 & 10 & 10 & 4.250614 & 1.249704 & 29.4 \\
  \hline
 \end{tabular}
 \caption{Результаты проверки быстродействия решения}
 \label{tab:prof}
\end{table}

В самом худшем случае на анализ одного обмена (с учётом накладных расходов) в 
среднем приходится $\hat{t_{a}} = t_{total} / N_{exch} \approx 3.9 * 10^{-5}$ 
с. 

В ходе исследования были изучены трассы обменов на реальных распределённых 
системах. Используем данные, полученные при изучении одной из этих трасс, для 
оценки возможности использования анализатора при регистрации обменов в реальном 
времени.

Средняя длительность обмена в трассе обмена, записанной на реальной системе: 
$\hat{t_{ex}} \approx 3.6 * 10^{-4}$ c. При длительности записанной трассы 
$t_{T} = 50$ с и количестве зарегистрированных обменов $\hat{N_{exch}} = 28452$ 
получаем среднюю продолжительность паузы между обменами:

$$
\hat{p} = \frac{t_{T} - \hat{t_{ex}} * \hat{N_{exch}}}{\hat{N_{exch}} - 1} 
\approx 1.39 * 10^{-3} c,
$$

что значительно больше средней оценки времени анализа одного обмена, полученной 
в результате проверки быстродействия в худшем случае:

$$
\hat{p} = 1.39 * 10^{-3} > 3.9 * 10^{-5} = \hat{t_{a}}.
$$

Также можно оценить максимальное количество обменов, обрабатываемых 
анализатором за секунду (с учётом накладных расходов) на предложенной 
конфигурации тестовой ЭВМ:

$$
v = 1 / \hat{t_{a}} \approx 2.56 * 10^{3}
$$

\subsubsection{Результаты}

Проверка быстродействия показала, что анализатор способен быстро обрабатывать 
достаточно большое количество ограничений и готов к работе как в режиме анализа 
записанной трассы, так и при регистрации обменов в реальном времени.