\section{Результаты апробации решения}

%TODO: описать, откуда взяты данные

\subsection{Конфигурация тестовой ЭВМ}
\label{subsec:test_stand}

\begin{itemize}
 \item Процессор: Intel Core i5-3320M, 2.6 ГГц;
 \item ОЗУ: 1024 МБ;
 \item ОС: Debian Linux 7 (wheezy), 32-разрядная;
 \item SSD. 
\end{itemize}

Тестирование решения проводилось на виртуальной ЭВМ в среде Oracle VM 
VirtualBox [\ref{virtualbox}].


\subsection{Генератор трасс}

Для проверки общей работоспособности решения без доступа к реальным 
или смоделированным ВС полезно провести апробацию средства на данных, 
полученных с помощью генератора трасс обменов.

Такой генератор был разработан ранее для демонстрационных целей. Для 
передачи данных агенту Анализатора МКИО используется специальное виртуальное 
устройство-петля. С помощью генератора можно получить последовательности 
обменов, подходящие для проверки возможностей анализатора ограничений.

Во время работы генератор непрерывно передаёт набор сообщений определённых 
типов и содержания с  частотой. 

Перечень генерируемых параметров приведён в таблице~\ref{tab:params}. Для 
каждого параметра приводится функция значения, а также описание 
обмена, передающего значение параметра (адреса и подадреса отправителя и 
получателя, формат сообщения, размер сообщения, номер первого слова, номер 
первого бита и длина битового поля). Параметры по умолчанию целочисленные.

Значение $i$ в определении функции - значение целочисленного счётчика, 
увеличиваемое на 1 каждый промежуток времени, заданный в исходном коде 
генератоа (по умолчанию - 300 мс).

Легенда таблицы:

\begin{itemize}
 \item $A_{src}$ - адрес источника сообщения (0-31, либо КК - контроллер 
канала);
 \item $SA_{src}$ - подадрес источника сообщения (0-30, либо КК - контроллер 
канала);
 \item $A_{dst}$ - адрес получателя сообщения (0-31, либо КК - контроллер 
канала);
 \item $SA_{dst}$ - подадрес получателя сообщения (0-30, либо КК - контроллер 
канала);
 \item $Sz_{msg}$ - размер сообщения (количество слов);
 \item $St_{word}$ - номер слова, в котором начинается битовое поле параметра;
 \item $St_{bit}$ - номер первого бита битового поля параметра в этом слове;
 \item $Sz_{bf}$ - количество бит в битовом поле параметра;
 \item $Sc$ - цена младшего бита битового поля.
\end{itemize}

\begin{table}[H]
\centering
\begin{tabular}{|l|l|*{9}{c|}}
\hline
 № & Функция & $A_{src}$ & $SA_{src}$ & $A_{dst}$ & $SA_{dst}$ & $Sz_{msg}$ & 
$St_{word}$ & $St_{bit}$ & $Sz_{bf}$ & $Sc$ \\
\hline
1 & $10^6 * sin(i / 10)$ & КК & КК & 2 & 1 & 4 & 0 & 0 & 32 & 1.0 \\
2 & $i$; 0x42 & КК & КК & 2 & 1 & 4 & 2 & 0 & 16 & 1.0 \\
3 & $10^6 * sin(i / 10)$ & КК & КК & 2 & 3 & 4 & 0 & 0 & 32 & 1.0 \\
4 & $i$; 0x42 & КК & КК & 2 & 3 & 4 & 2 & 0 & 16 & 1.0 \\
5 & $10^6 * sin(i / 10)$ & КК & КК & 2 & 4 & 4 & 0 & 0 & 32 & 1.0 \\
6 & $i$; 0x42 & КК & КК & 2 & 4 & 4 & 2 & 0 & 16 & 1.0 \\
\hline 
\end{tabular}
\caption{Перечень параметров, предоставляемых генератором}
\label{tab:params}
\end{table}

В значения некоторых параметров генератором вносятся ошибки. Ошибки 
для таких параметров описаны в таблице~\ref{tab:param_errors}. Частота 
возникновения здесь - количество циклов генератора, определяемых частотой 
увеличения внутреннего целочисленного счётчика $i$. $rnd()$ - значение, 
определяемое с помощью генератора псевдослучайных чисел (функции 
\textit{random()} стандартной библиотеки языка Си).

\begin{table}[H]
 \centering
 \begin{tabular}{|l|c|c|}
  \hline
  № & Ошибочное значение & Частота возникновения \\
  \hline
  5 & побитовое НЕ верного значения & $rnd()$ \\
  6 & 0xDEAD & 10 \\
  \hline
 \end{tabular}
 \caption{Ошибочные значения параметров}
 \label{tab:param_errors}
\end{table}

\subsection{Примеры описаний протоколов и апробация}

Апробация возможностей анализатора ограничений проводилась на наборе примеров 
описаний протоколов и ограничений, соответствующих трассам обменов, полученным 
с помощью генератора.

Ниже приводятся примеры описаний протоколов, графики значений параметров, 
описанных в этих протоколах и результаты работы анализатора для этих 
описаний протоколов. Проверка работы анализатора проводилась в режиме 
регистрации обменов в реальном времени.

\subsubsection{Пример 1}

В примере 1 наблюдается изменение значений младших 8 бит значений параметров 4 
и 6 из таблицы~\ref{tab:params} (назовём полученные параметры 4а и 6а 
соответственно). Значение параметра 4а 
постоянно и равно 42. Если для параметра 6а не выполнено условие генерации 
ошибки, его значение равно значению параметра 4а, иначе - константе 173 (0xAD). 

Параметры связываются в группу с идентификатором \textit{int42} со 
значениями погрешности по умолчанию (5\%). Поскольку параметр 6а принимает 
ошибочные значения со значительным отклонением (больше 5\%), в отчёте 
анализатора должны появиться сообщения о расхождении значений параметров в 
группе \textit{int42}.

График изменения значений параметров 
приводится на рисунке~\ref{fig:param_bind_graph}. Отчёт анализатора приводится 
на рисунке~\ref{fig:param_bind_report}.

\lstinputlisting[caption=Пример описания протокола 
1]{tests/param_bind/protocol.xml}

\begin{figure}[H]
 \centering
 \includegraphics[scale=0.6]{tests/param_bind/graph}
 \caption{График изменения значений параметров (красный - 4а, зелёный - 6а)}
 \label{fig:param_bind_graph}
\end{figure}

\begin{figure}[H]
 \centering
 \includegraphics[scale=0.4]{tests/param_bind/report}
 \caption{Результат работы анализатора}
 \label{fig:param_bind_report}
\end{figure}

\subsubsection{Пример 2}

В примере 2 наблюдается изменение значения параметра 6 из 
таблицы~\ref{tab:params}. Для проверки факта равенства параметра ошибочному 
значению 57005 (0xDEAD) используется ограничение error\_value.

График изменения значения параметра 
приводится на рисунке~\ref{fig:param_error_value_graph}. Отчёт анализатора 
приводится на рисунке~\ref{fig:param_error_value_report}.

\lstinputlisting[caption=Пример описания протокола 
2]{tests/param_error_value/protocol.xml}

\begin{figure}[H]
 \centering
 \includegraphics[scale=0.6]{tests/param_error_value/graph}
 \caption{График изменения значений параметра 6}
 \label{fig:param_error_value_graph}
\end{figure}

\begin{figure}[H]
 \centering
 \includegraphics[scale=0.4]{tests/param_error_value/report}
 \caption{Результат работы анализатора}
 \label{fig:param_error_value_report}
\end{figure}

\subsubsection{Пример 3}

В примере 3 наблюдается изменение значения параметра 6 из 
таблицы~\ref{tab:params}. Для обнаружения ``выбросов'' (значительных (>5\%)
отклонений значения параметра от исходной функции) используется ограничение 
smooth с максимальным значением производной по модулю, равным 1000 (значение 
подобрано эмпирически).

График изменения значения параметра 
приводится на рисунке~\ref{fig:param_smooth_graph}. Отчёт анализатора 
приводится на рисунке~\ref{fig:param_smooth_report}.

\lstinputlisting[caption=Пример описания протокола 
3]{tests/param_smooth/protocol.xml}

\begin{figure}[H]
 \centering
 \includegraphics[scale=0.6]{tests/param_smooth/graph}
 \caption{График изменения значений параметра 6}
 \label{fig:param_smooth_graph}
\end{figure}

\begin{figure}[H]
 \centering
 \includegraphics[scale=0.4]{tests/param_smooth/report}
 \caption{Результат работы анализатора}
 \label{fig:param_smooth_report}
\end{figure}

\subsection{Нагрузочное тестирование}

В рамках нагрузочного тестирования проверяется быстродействие анализатора.

Для проведения нагрузочного тестирования был модифицирован генератор обменов, 
использованный для проведения предыдущих тестов. Целью модификации было 
получение большего числа типов сообщений для проверки производительности 
анализатора.

Для проведения тестирования записана трасса обменов от 
модифицированного генератора длительностью 20 минут, содержащая 108873 обмена. 

Файл протокола содержит описания всех сообщений, получаемых от 
модифицированного генератора (всего 250 типов сообщений). Количество типов 
сообщений выбрано в соответствии с таковым в изученных трассах обменов 
используемых РВС. Для каждого из типов сообщений описываются ограничения; 
количество ограничений для каждого из проведённых тестов приводится в 
таблице~\ref{tab:prof}.

Для типов сообщений, подразумевающих передачу полезной нагрузки, описаны также 
передаваемые параметры и ограничения для них. Количество параметров в протоколе 
и количество ограничений для каждого из параметров приведены в 
таблице~\ref{tab:prof}.

Тестирование проводилось для разного количества описаний сообщений и 
параметров, а также разного количества ограничений на сообщения и параметры.

Проект собирался с выставленными флагами оптимизации (перед сборкой была 
создана глобальная переменная окружения CL\_OPTIMIZE=1, используемая системой 
сборки cvslvk [\ref{cvslvk}]).

Для каждого тестового протокола проводилась операция загрузки трассы. Операция 
проводилась 5 раз для каждого тестового протокола, после чего подсчитывалось 
среднее значение времени загрузки трассы и среднее значение времени, 
потраченного в контексте анализатора. Время в контексте анализатора - время, 
потраченное в методах-обработчиках:
\begin{itemize}
 \item AnalyzerCore::onParamValueUpdated(),
 \item AnalyzerCore::onAddExchange(). 
\end{itemize}

Время загрузки трассы оценивается временем выполнения метода 

\begin{itemize}
 \item ExchangeContainer::addSubscriber().
\end{itemize}

Результаты тестирования приведены в таблице~\ref{tab:prof}.

Легенда таблицы:

\begin{itemize}
 \item $N_{msgs}$ - общее число сообщений, описанных в файле протокола;
 \item $N_{params}$ - общее число параметров, описанных в файле протокола;
 \item $M_{msg}$ - количество ограничений на каждое из описанных сообщений;
 \item $M_{param}$ - количество ограничений на каждый из описанных параметров;
 \item $t_{total}$ - среднее значение оценки времени загрузки трассы с данным 
описанием протокола (в секундах);
 \item $t_{a}$ - среднее значение оценки времени, потраченного в контексте 
анализатора при загрузке трассы с данным описанием протокола (в секундах).
\end{itemize}


\begin{table}[H]
 \centering
 \begin{tabular}{|c|*{6}{c|}}
  \hline
  № теста & $N_{msgs}$ & $N_{params}$ & $M_{msg}$ & $M_{param}$ & $t_{total}$, 
с & $t_{a}$, с \\
  \hline
  1 & 128 & 128 & 5 & 5 & 3.610892 & 0.572862 \\
  2 & 128 & 240 & 5 & 5 & 0.0 & 0.0 \\
  3 & 240 & 128 & 5 & 5 & 0.0 & 0.0 \\
  4 & 240 & 240 & 5 & 5 & 0.0 & 0.0 \\
  5 & 128 & 128 & 10 & 10 & 0.0 & 0.0 \\
  \hline
 \end{tabular}
 \caption{Результаты нагрузочного тестирования решения}
 \label{tab:prof}
\end{table}


\subsection{Результаты}

В процессе апробации было выяснено, что анализатор корректно обрабатывает 
ограничения, накладываемые на обмены и параметры в описании протокола.

В планах до защиты ВКР есть также апробация анализатора на трассах с 
известными проблемами, зарегистрированных в ходе испытаний на стенде 
[\ref{stand}].