\section{Результаты апробации решения}

%TODO: описать, откуда взяты данные

\subsection{Конфигурация тестовой ЭВМ}
\label{subsec:test_stand}

\begin{itemize}
 \item Процессор: Intel Core i5-3320M, 2.6 ГГц;
 \item ОЗУ: 1024 МБ;
 \item ОС: Debian Linux 7 (wheezy), 32-разрядная;
 \item SSD. 
\end{itemize}

Тестирование решения проводилось на виртуальной ЭВМ в среде Oracle VM 
VirtualBox [\ref{virtualbox}].

\subsection{Примеры описаний протоколов и апробация}

Апробация возможностей анализатора ограничений проводилась на наборе примеров 
описаний протоколов и ограничений, соответствующих трассам обменов, полученным 
с помощью генератора.

\subsubsection{Пример 1}

В примере 1 наблюдается изменение значений младших 8 бит значений параметров 4 
и 6 из таблицы~\ref{tab:params} (назовём полученные параметры 4а и 6а 
соответственно). Значение параметра 4а 
постоянно и равно 42. Если для параметра 6а не выполнено условие генерации 
ошибки, его значение равно значению параметра 4а, иначе - константе 173 (0xAD). 

Параметры связываются в группу с идентификатором \textit{int42} со 
значениями погрешности по умолчанию (5\%). Поскольку параметр 6а принимает 
ошибочные значения со значительным отклонением (больше 5\%), в отчёте 
анализатора должны появиться сообщения о расхождении значений параметров в 
группе \textit{int42}.

График изменения значений параметров 
приводится на рисунке~\ref{fig:param_bind_graph}. Отчёт анализатора приводится 
на рисунке~\ref{fig:param_bind_report}.

\subsubsection{Пример 2}

В примере 2 наблюдается изменение значения параметра 6 из 
таблицы~\ref{tab:params}. Для проверки факта равенства параметра ошибочному 
значению 57005 (0xDEAD) используется ограничение error\_value.

График изменения значения параметра 
приводится на рисунке~\ref{fig:param_error_value_graph}. Отчёт анализатора 
приводится на рисунке~\ref{fig:param_error_value_report}.

\subsubsection{Пример 3}

В примере 3 наблюдается изменение значения параметра 6 из 
таблицы~\ref{tab:params}. Для обнаружения ``выбросов'' (значительных (>5\%)
отклонений значения параметра от исходной функции) используется ограничение 
smooth с максимальным значением производной по модулю, равным 1000 (значение 
подобрано эмпирически).

График изменения значения параметра 
приводится на рисунке~\ref{fig:param_smooth_graph}. Отчёт анализатора 
приводится на рисунке~\ref{fig:param_smooth_report}.

\subsection{Нагрузочное тестирование}

В рамках нагрузочного тестирования проверяется быстродействие анализатора.

Для проведения нагрузочного тестирования был модифицирован генератор обменов, 
использованный для проведения предыдущих тестов. Целью модификации было 
получение большего числа типов сообщений для проверки производительности 
анализатора.

Для проведения тестирования записана трасса обменов от 
модифицированного генератора длительностью 20 минут, содержащая $N_{exch} 
= 108873$ обмена. 

Файл протокола содержит описания всех сообщений, получаемых от 
модифицированного генератора (всего 250 типов сообщений). Количество типов 
сообщений выбрано в соответствии с таковым в изученных трассах обменов 
используемых РВС. Для каждого из типов сообщений описываются ограничения; 
количество ограничений для каждого из проведённых тестов приводится в 
таблице~\ref{tab:prof}.

Для типов сообщений, подразумевающих передачу полезной нагрузки, описаны также 
передаваемые параметры и ограничения для них. Количество параметров в протоколе 
и количество ограничений для каждого из параметров приведены в 
таблице~\ref{tab:prof}.

Тестирование проводилось для разного количества описаний сообщений и 
параметров, а также разного количества ограничений на сообщения и параметры.

Проект собирался с выставленными флагами оптимизации (перед сборкой была 
создана глобальная переменная окружения CL\_OPTIMIZE=1, используемая системой 
сборки cvslvk [\ref{cvslvk}]).

Для каждого тестового протокола проводилась операция загрузки трассы. Операция 
проводилась 5 раз для каждого тестового протокола, после чего подсчитывалось 
среднее значение времени загрузки трассы и среднее значение времени, 
потраченного в контексте анализатора. Время в контексте анализатора - время, 
потраченное в методах-обработчиках:
\begin{itemize}
 \item AnalyzerCore::onParamValueUpdated(),
 \item AnalyzerCore::onAddExchange(). 
\end{itemize}

Время загрузки трассы оценивается временем выполнения метода 

\begin{itemize}
 \item ExchangeContainer::addSubscriber().
\end{itemize}

Для 5 независимых запусков среднеквадратичное отклонение результатов 
измерения оценки времени выполнения для обоих случаев не превысило 2.5\%.

Результаты тестирования приведены в таблице~\ref{tab:prof}.

Легенда таблицы:

\begin{itemize}
 \item $N_{msgs}$ - общее число сообщений, описанных в файле протокола;
 \item $N_{params}$ - общее число параметров, описанных в файле протокола;
 \item $M_{msg}$ - количество ограничений на каждое из описанных сообщений;
 \item $M_{param}$ - количество ограничений на каждый из описанных параметров;
 \item $t_{total}$ - среднее значение оценки времени загрузки трассы с данным 
описанием протокола (в секундах);
 \item $t_{a}$ - среднее значение оценки времени, потраченного в контексте 
анализатора при загрузке трассы с данным описанием протокола (в секундах);
 \item $\%$ - доля времени, потраченного в контексте анализатора, по 
отношению к общему времени загрузки трассы (в процентах, для средних значений 
оценки времени).
\end{itemize}


\begin{table}[H]
 \centering
 \begin{tabular}{|c|*{7}{c|}}
  \hline
  № теста & $N_{msgs}$ & $N_{params}$ & $M_{msg}$ & $M_{param}$ & $t_{total}$, 
с & $t_{a}$, с & \% \\
  \hline
  1 & 128 & 128 & 5   & 5   & 3.610892 & 0.572862 & 15.86 \\
  2 & 128 & 240 & 5   & 5   & 5.470608 & 0.888364 & 16.23 \\
  3 & 240 & 128 & 5   & 5   & 3.517034 & 0.553243 & 15.73 \\
  4 & 240 & 240 & 5   & 5   & 5.459458 & 0.889838 & 16.29 \\
  5 & 128 & 128 & 10 & 10 & 4.250614 & 1.249704 & 29.4 \\
  \hline
 \end{tabular}
 \caption{Результаты нагрузочного тестирования решения}
 \label{tab:prof}
\end{table}

В самом худшем случае на анализ одного обмена (с учётом накладных расходов) в 
среднем приходится $\hat{t_{a}} = t_{total} / N_{exch} \approx 3.9 * 10^{-5}$ 
с. 

В ходе исследования были изучены трассы обменов на реальных распределённых 
системах. Используем данные, полученные при изучении одной из этих трасс, для 
оценки возможности использования анализатора при регистрации обменов в реальном 
времени.

Средняя длительность обмена в трассе обмена, записанной на реальной системе: 
$\hat{t_{ex}} \approx 3.6 * 10^{-4}$ c. При длительности записанной трассы 
$t_{T} = 50$ с и количестве зарегистрированных обменов $\hat{N_{exch}} = 28452$ 
получаем среднюю продолжительность паузы между обменами:

$$
\hat{p} = \frac{t_{T} - \hat{t_{ex}} * \hat{N_{exch}}}{\hat{N_{exch}} - 1} 
\approx 1.39 * 10^{-3} c,
$$

что значительно больше средней оценки времени анализа одного обмена, полученной 
в результате нагрузочного тестирования в худшем случае:

$$
\hat{p} = 1.39 * 10^{-3} > 3.9 * 10^{-5} = \hat{t_{a}}.
$$

Таким образом, тестирование показывает, что быстродействие анализатора 
позволяет использовать его при регистрации обменов в реальном времени.

\subsection{Результаты}

В процессе апробации было выяснено, что анализатор корректно обрабатывает 
ограничения, накладываемые на обмены и параметры в описании протокола. Также 
нагрузочное тестирование показало, что анализатор способен обрабатывать 
достаточно большое количество ограничений в режиме анализа записанной трассы 
и при регистрации обменов в реальном времени.