\section{Результаты апробации решения}

%TODO: описать, откуда взяты данные

\subsection{Генератор трасс}

Для проверки общей работоспособности решения без доступа к реальным 
или смоделированным ВС полезно провести апробацию средства на данных, 
полученных с помощью генератора трасс обменов.

Такой генератор был разработан ранее для демонстрационных целей. Для 
передачи данных агенту Анализатора МКИО используется специальное виртуальное 
устройство-петля. С помощью генератора можно получить последовательности 
обменов, подходящие для проверки возможностей анализатора ограничений.

Во время работы генератор непрерывно передаёт набор сообщений определённых 
типов и содержания с  частотой. 

Перечень генерируемых параметров приведён в таблице~\ref{tab:params}. Для 
каждого параметра приводится функция значения, а также описание 
обмена, передающего значение параметра (адреса и подадреса отправителя и 
получателя, формат сообщения, размер сообщения, номер первого слова, номер 
первого бита и длина битового поля). Параметры по умолчанию целочисленные.

Значение $i$ в определении функции - значение целочисленного счётчика, 
увеличиваемое на 1 каждые 300 мс.

Легенда таблицы:

\begin{itemize}
 \item $A_{src}$ - адрес источника сообщения (0-31, либо КК - контроллер 
канала);
 \item $SA_{src}$ - подадрес источника сообщения (0-30, либо КК - контроллер 
канала);
 \item $A_{dst}$ - адрес получателя сообщения (0-31, либо КК - контроллер 
канала);
 \item $SA_{dst}$ - подадрес получателя сообщения (0-30, либо КК - контроллер 
канала);
 \item $Sz_{msg}$ - размер сообщения (количество слов);
 \item $St_{word}$ - номер слова, в котором начинается битовое поле параметра;
 \item $St_{bit}$ - номер первого бита битового поля параметра в этом слове;
 \item $Sz_{bf}$ - количество бит в битовом поле параметра;
 \item $Sc$ - цена младшего бита битового поля.
\end{itemize}

\begin{table}[H]
\centering
\begin{tabular}{|l|l|*{9}{c|}}
\hline
 № & Функция & $A_{src}$ & $SA_{src}$ & $A_{dst}$ & $SA_{dst}$ & $Sz_{msg}$ & 
$St_{word}$ & $St_{bit}$ & $Sz_{bf}$ & $Sc$ \\
\hline
1 & $10^6 * sin(i / 10)$ & КК & КК & 2 & 1 & 4 & 0 & 0 & 32 & 1.0 \\
2 & $i$; 0x42 & КК & КК & 2 & 1 & 4 & 2 & 0 & 16 & 1.0 \\
3 & $10^6 * sin(i / 10)$ & КК & КК & 2 & 3 & 4 & 0 & 0 & 32 & 1.0 \\
4 & $i$; 0x42 & КК & КК & 2 & 3 & 4 & 2 & 0 & 16 & 1.0 \\
5 & $10^6 * sin(i / 10)$ & КК & КК & 2 & 4 & 4 & 0 & 0 & 32 & 1.0 \\
6 & $i$; 0x42 & КК & КК & 2 & 4 & 4 & 2 & 0 & 16 & 1.0 \\
\hline 
\end{tabular}
\caption{Перечень параметров, предоставляемых генератором}
\label{tab:params}
\end{table}

В значения некоторых параметров генератором вносятся ошибки. Ошибки 
для таких параметров описаны в таблице~\ref{tab:param_errors}. Частота 
возникновения здесь - количество циклов генератора, определяемых частотой 
увеличения внутреннего целочисленного счётчика $i$. $rnd()$ - значение, 
определяемое с помощью генератора псевдослучайных чисел (функции 
\textit{random()} стандартной библиотеки языка Си).

\begin{table}[H]
 \centering
 \begin{tabular}{|l|c|c|}
  \hline
  № & Ошибочное значение & Частота возникновения \\
  \hline
  5 & побитовое НЕ верного значения & $rnd()$ \\
  6 & 0xDEAD & 10 \\
  \hline
 \end{tabular}
 \caption{Ошибочные значения параметров}
 \label{tab:param_errors}
\end{table}




