\section{Описание реализации}

В данном разделе приводится краткое описание реализации решения на языке С++ с 
использованием возможностей инструментария Qt и его библиотеки для языка С++, а 
также с использованием инструментария, реализованного в рамках tabexchange.

\subsection{Внутреннее представление ограничения}

В качестве внутреннего представления отдельных ограничений используются 
специальные структуры: 

\begin{itemize}
 \item ExchangeRestrictDescription,
 \item ParamRestrictDescription,
\end{itemize}

наследующие тип RescrictDescription. Структуры содержат поля для хранения 
информации согласно описанию внутреннего представления (см. 
раздел~\ref{subsec:inner}) с дополнительным полем для идентификатора объекта - 
текстового идентификатора обмена (параметра).

Дополнительно в рамках этих структур реализованы методы для сохранения и 
загрузки полей из контейнера QSettings, а также метод для загрузки 
значений полей при импорте XML-описания протокола.

Структуры определены в файлах tabexchange/analyzercore.[h|cpp].

\subsection{Контейнеры для описания ограничений}

Для хранения внутреннего представления описания ограничений были реализованы 
контейнеры
\begin{itemize}
 \item ExchangeRestrictionContainer,
 \item ParamRestrictionContainer.
\end{itemize}

Контейнеры реализованы на базе класса NamedObjectContainer, используемого в 
tabexchange для реализации контейнеров для именованных объектов с возможностью 
добавления, использования, редактирования (с оповещением классов-пользователей 
контейнера) объектов, а также автоматического сохранения содержимого контейнера 
в пользовательский репозиторий Анализатора МКИО и последующей загрузки.

Именами в контейнерах служат текстовые идентификаторы: для обменов - 
идентификаторы сообщений, совпадающие с таковыми в контейнере описаний 
сообщений MessageContainer, для параметров - пара ``идентификатор сообщения + 
идентификатор параметра'', позволяющая также найти описание параметра в 
контейнере описания параметров ParameterContainer.

В качестве хранимых объектов используются массивы описаний ограничений

\subsection{Импорт описания ограничений}

За импорт файла описания протокола в tabexchange отвечают следующие классы:

\begin{itemize}
 \item BDPivImporter (файл tabexchange/mppexportimport.cpp) - класс, 
реализующий разбор XML-файла описания протокола преобразованием в представление 
QSettings, используемое в Qt-приложениях для хранения конфигурации в виде пар 
``ключ-значение'' с иерархией ключей. Объект класса заворачивается в функцию 
read(), передаваемую QSettings::registerFormat() для обобщённой обработки 
конфигурационных файлов различных форматов;.
 \item MPPExportImportModel, MPPExportImportDialog (файлы 
mppexportimport.cpp, mppexportimport.h в директории tabexchange) - классы, 
реализующие диалоговое окно импорта конфигурационного файла с возможностью 
разрешения конфликтов импорта.
\end{itemize}

В реализации анализатора ограничений были внесены следующие изменения:

\begin{itemize}
\item в класс BDPivImporter добавлена обработка XML-тегов <restrict> и 
<restricts>;
\item в метод MPPExportImportDialog::accept() добавлено заполнение контейнеров 
ExchangeRestrictionContainer и ParamRestrictionContainer в соответствии с 
описанием ограничений в загружаемом файле описания протокола.
\end{itemize}

