\section{Введение}

\subsection{Описание проблемы и актуальность}

Предметом исследования данной работы являются распределённые вычислительные 
системы (РВС) управления сложными техническими объектами (кораблями, 
самолётами, производственными линиями и т.п.). \textit{Распределённая 
вычислительная система} - это набор устройств (вычислительных машин, датчиков и 
актуаторов), объединённых с помощью каналов передачи данных (одного или 
нескольких). 

Как правило, в таких системах компоненты взаимодействуют друг с другом в 
соответствии с набором протоколов. \textit{Протокол} - документ, содержащий 
информацию о компонентах системы, адресации на каналах передачи данных, наборе 
передаваемых сообщений, их свойствах (форматах, частотах, ограничениях на 
последовательность и т.п.), а также описание содержимого этих сообщений. 
Протоколы согласуются и утверждаются разработчиками РВС.

Одной из задач интеграционного тестирования РВС является проверка соблюдения 
компонентами требований, описанных в этих протоколах. Стоит отметить, 
что часто протоколы представляют собой неформальное текстовое описание, 
непригодное для проведения автоматической проверки. Тем не менее, из 
используемых на практике протоколов можно выделить общий набор типов требований, 
предъявляемых к сообщениям и передаваемым в них данным. После предложения 
формального представления этих требований появляется возможность проведения 
автоматического анализа системы на их соблюдение.

Несоблюдение системой требований из протоколов может означать, что при 
проектировании или разработке РВС была допущена ошибка, или какое-либо 
устройство (или несколько устройств) в системе работают некорректно.

Поскольку в состав РВС может входить большое количество устройств (десятки или 
сотни), количество требований может оказаться таким, что проверка системы на 
соответствие им без автоматического средства окажется очень сложной или 
невозможной, что подтверждает актуальность задачи \textit{автоматического} 
анализа системы.

Таким образом, при разработке и отладке РВС полезно иметь инструмент для 
проведения автоматического анализа системы на соответствие требованиям, 
предъявляемым в протоколах.

\subsection{Анализатор МКИО}

В ЛВК разрабатывается инструмент для работы с каналами связи РВС -  
Анализатор МКИО [\ref{opermon_reg}]. Инструмент поддерживает работу с  
различными типами каналов связи, в том числе MIL STD-1553B / МКИО ГОСТ Р 
52070-2003 [\ref{gost_r_52070-2003}], Fibre Channel, CAN и ARINC 429. 
Анализатор МКИО реализован на языке C++ с использованием инструментария Qt 4 
[\ref{qt_ref}].

Среди возможностей инструмента следует отметить следующие:

\begin{itemize}
 \item регистрация обменов в реальном времени;
 \item запись трасс обменов;
 \item получение значений отдельных параметров, передаваемых в сообщениях;
 \item построение графиков значений параметров.
\end{itemize}

Здесь \textit{трасса} - последовательность зарегистрированных сообщений; 
\textit{параметр} - логическая единица передаваемой информации, которой 
соответствует набор атомарных значений из полезной нагрузки сообщений. Примеры 
параметров для бортовой РВС: скорость, высота над уровнем моря, координаты GPS.

Правила для вычисления значений параметров предлагаются пользователем 
Анализатора МКИО.

Анализатор МКИО используется при разработке и отладке бортовых РВС. Однако, 
актуальная версия Анализатора МКИО не поддерживает автоматическую проверку 
требований протоколов. 

В рамках данной работы будет предложена реализация средства автоматической 
проверки требований протоколов на основе инструмента Анализатор МКИО.