\section{Введение}

\subsection{Описание проблемы и актуальность}

%TODO: что делать с термином?
Предметом исследования данной работы являются распределённые 
информационно-управляющие системы реального времени [\ref{vius_rv_course}]. 
Далее в данной работе будет использоваться термин ``распределённая 
вычислительная система'' (РВС).

При разработке и отладке РВС возникает задача проверки корректности 
функционирования системы. Для решения этой задачи используются следующие 
подходы:

\begin{itemize}
 \item \textit{тестирование} - подразумевает вмешательство в работу системы. 
Обычно для этого в каналы связи отправляются управляющие сообщения, 
после чего анализируется отклик системы. Этот подход часто используется для 
проверки отдельных узлов, при этом управляющие сообщения моделируют 
процесс функционирования целой системы;
 \item \textit{мониторинг} - без вмешательства в работу системы. Проводится 
запись \textit{трассы} обменов - последовательность сообщений и пауз, 
наблюдаемых в каналах связи функционирующей системы, после чего проводится 
анализ трассы на предмет некорректной работы узлов.
\end{itemize}

В обоих подходах после сбора трасс обменов требуется провести их анализ. Это 
можно сделать следующими способами:

\begin{itemize}
 \item сравнением записанной трассы с эталоном;
 \item проверкой соблюдения трассой набора требований.
\end{itemize}

Решение задачи сравнения трассы обменов с эталоном предложено, к примеру, в 
работе [\ref{gordeev_diploma}]. Однако, например, при анализе корректности 
функционирования сложной системы (состоящей из двух и более устройств), 
задача подготовки эталонной трассы обменов может оказаться неоправданно 
сложной. В этом случае целесообразно проводить поверку трассы 
информационного обмена на соответствие набору требований.

Как правило, логика функционирования РВС и описание обмена данными между 
компонентами оформлена в виде набора утверждённых 
разработчиками РВС \textit{протоколов}, часто представляющих собой неформальное 
текстовое описание, непригодное для проведения автоматической проверки.
%\textit{Протокол} - документ, содержащий 
%информацию о компонентах системы, адресации на каналах передачи данных, наборе 
%передаваемых сообщений, их свойствах (форматах, частотах, ограничениях на 
%последовательность и т.п.), а также описание содержимого этих сообщений. 
Таким образом, возникает необходимость в \textit{формализации} описания 
требований.

% TODO: про десятки - ссылка на Балашова стр.5
Поскольку в состав РВС может входить большое количество устройств (десятки), 
количество проверяемых требований может оказаться таким, что анализ 
системы на соответствие им без автоматического средства окажется очень сложным 
или невозможным, что подтверждает актуальность задачи \textit{автоматического} 
анализа системы.

Таким образом, при разработке и отладке РВС полезно иметь инструмент для 
проведения автоматического анализа системы на соответствие требованиям, 
предъявляемым в протоколах.

Проверку требований можно проводить в рамках функционального тестирования 
бортового комплекса. Однако, это требует ручного написания тестовых процедур 
для каждого проверяемого требования с использованием императивной семантики.

\subsection{Анализатор MIL STD-1553B}

В ЛВК разрабатывается семейство инструментов для мониторинга информационного 
обмена на каналах связи РВС. Один из них - Анализатор MIL STD-1553B 
[\ref{opermon_reg}], используемый для работы с канами связи MIL STD-1553B 
(в России определённый ГОСТ Р 52070-2003 [\ref{gost_r_52070-2003}]). Средство 
реализовано на языке C++ с использованием инструментария Qt 4 [\ref{qt_ref}].

Среди возможностей инструмента следует отметить следующие:

\begin{itemize}
 \item регистрация обменов в реальном времени;
 \item запись трасс обменов;
 \item получение значений отдельных параметров, передаваемых в сообщениях;
 \item построение графиков значений параметров.
\end{itemize}

% TODO: взять определение откуда-нибудь из изввестных источников
Здесь \textit{трасса} - последовательность зарегистрированных сообщений и пауз; 
\textit{параметр} - логическая единица передаваемой информации, которой 
соответствует набор атомарных значений из полезной нагрузки сообщений. Примеры 
параметров для бортовой РВС: скорость, высота над уровнем моря, координаты GPS.

%Правила для вычисления значений параметров предлагаются пользователем 
%Анализатора МКИО.

Анализатор MIL STD-1553B используется при разработке и отладке бортовых РВС. 
Однако, актуальная версия Анализатора не поддерживает функциональность 
автоматической поверки информационного обмена на соответствие требованиям.

В рамках данной работы будет предложена реализация средства автоматической 
проверки требований на основе инструмента Анализатор MIL STD-1553B.