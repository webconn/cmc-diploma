\section{Введение}

\subsection{Задача анализа корректности работы РВС}

Предметом исследования данной работы являются распределённые вычислительные 
системы (РВС). \textit{Распределённая вычислительная система} - это 
набор независимых вычислительных устройств, представляющихся их 
пользователям единой объединённой системой [\ref{tanenbaum}]. Частным случаем 
РВС, рассматриваемым в данной работе, являются распределённые 
информационно-управляющие системы реального времени [\ref{vius_rv_course}].

При разработке и отладке РВС возникает задача проверки корректности 
функционирования системы. Для решения этой задачи используются следующие 
подходы [\ref{vius_rv_course}]:

\begin{itemize}
 \item \textit{тестирование} - исследование системы с вмешательством в работу.
Обычно для этого в каналы связи отправляются управляющие сообщения, 
после чего анализируется отклик системы. Этот подход часто используется для 
проверки отдельных узлов, при этом управляющие сообщения моделируют 
процесс функционирования целой системы;
 \item \textit{мониторинг} - исследование без вмешательства в работу системы. 
Проводится запись \textit{трассы} обменов - последовательности сообщений и 
пауз, наблюдаемых в каналах связи функционирующей системы, после чего 
проводится анализ трассы на предмет некорректной работы узлов.
\end{itemize}

После сбора трасс обменов требуется провести их анализ. Это можно сделать 
следующими способами:

\begin{itemize}
 \item сравнением записанной трассы с эталоном. Решение этой задачи предложено, 
к примеру, в работе [\ref{gordeev_diploma}]. Но при анализе корректности 
функционирования достаточно сложной системы (состоящей из многих 
устройств), задача подготовки эталонной трассы обменов может оказаться 
неоправданно сложной;
 \item проверкой соблюдения трассой набора требований. В этом случае 
проверяется не вся трасса целиком, а некоторые её характеристики (свойства 
обменов, значения передаваемых параметров и т.п.). При таком подходе упрощается 
процесс подготовки исходных данных для проведения проверки (описания отдельных 
требований имеют достаточно простые формулировки), а также появляется 
возможность анализировать работу отдельных узлов системы вне зависимости от 
наличия на линии абонентов, не участвующих в проверке.
\end{itemize}

Как правило, логика функционирования РВС и описание обмена данными между 
компонентами оформлены в виде набора утверждённых 
разработчиками РВС \textit{протоколов}, часто представляющих собой неформальное 
текстовое описание, непригодное для проведения автоматической проверки.
Таким образом, возникает необходимость в \textit{формализации} описания 
требований.

Поскольку в состав РВС могут входить десятки различных устройств, 
количество проверяемых требований может оказаться таким, что анализ 
системы на соответствие им без автоматического средства окажется очень сложным 
или невозможным, что подтверждает актуальность задачи \textit{автоматического} 
анализа системы.

Таким образом, при разработке и отладке РВС полезно иметь инструмент для 
проведения автоматического анализа системы на соответствие требованиям, 
предъявляемым в протоколах.

Проверку требований можно проводить в рамках функционального тестирования 
бортового комплекса. Однако, это требует ручного написания тестовых процедур 
для каждого проверяемого требования с использованием императивной семантики, 
что достаточно трудоёмко. Более того, анализ требований в рамках 
функционального тестирования будет достаточно ресурсоёмким процессом в силу 
особенностей инструментария [\ref{stand3}].

\subsection{Анализатор информационного обмена}

В Лаборатории вычислительных комплексов факультета ВМК МГУ разрабатывается 
семейство инструментов для мониторинга и анализа информационного обмена на 
каналах связи РВС. Эти средства активно используются при разработке и 
отладке бортовых РВС для самолётов и кораблей на различных этапах жизненного 
цикла систем, в том числе при стендовом моделировании [\ref{stand3}]. На 
сегодняшний день инструментарий нашёл применение в разработках таких 
организаций, как ``ОКБ Сухого'' и ``ОАК - Центр Комплексирования''.

%Один из них - Анализатор MIL STD-1553B [\ref{opermon_reg}], используемый 
%для работы с канами связи MIL STD-1553B (в России определённый ГОСТ Р 
%52070-2003 [\ref{gost_r_52070-2003}]). Средство реализовано на языке C++ с 
%использованием инструментария Qt 4 [\ref{qt_ref}].

%MIL STD-1553B [\ref{mils}] [\ref{gost_r_52070}], 

Среди возможностей средств следует отметить следующие:

\begin{itemize}
 \item регистрация обменов в реальном времени;
 \item запись трасс обменов;
 \item получение значений отдельных параметров, передаваемых в сообщениях. Для 
бортовых РВС примерами параметров являются скорость, высота над уровнем моря, 
координаты GPS и т.п.
\end{itemize}

Данное ПО реализовано на языке С++ с использованием инструментария Qt 4 
[\ref{qt_ref}]. Средства имеют графический интерфейс пользователя, 
поддерживающий отображение обменов и параметров в табличном виде с возможностью 
поиска и фильтрации, а также построение графиков значений параметров.

% TODO: взять определение откуда-нибудь из изввестных источников
%Здесь \textit{трасса} - последовательность зарегистрированных сообщений и 
%пауз; 
%\textit{параметр} - логическая единица передаваемой информации, которой 
%соответствует набор атомарных значений из полезной нагрузки сообщений. Примеры 
%параметров для бортовой РВС: .

В то же время, разработанный инструментарий до сих пор не поддерживает 
функциональность автоматической поверки информационного обмена на соответствие 
требованиям.

Стоит отметить, что для различных каналов связи существуют особенности, 
касающиеся структуры обменов. К примеру, на разных каналах могут использоваться 
специфические флаги ошибок, не имеющие прямых аналогов в других стандартах. 
Также может различаться природа передаваемых данных (например, в бортовых РВС 
самолётов используются отдельные каналы для обмена показаниями датчиков и для 
передачи видеопотоков). Это значит, что к обменам на разных каналах могут 
предъявляться несовместимые наборы требований.

Одним из типов каналов, поддерживаемых в инструментарии, является MIL STD-1553B 
[\ref{gost_r_52070-2003}][\ref{mils}] - канал с централизованным управлением, 
активно используемый в бортовых РВС для организации обмена данными произвольных 
типов.

В рамках данной работы предлагается расширение функциональности разработанного 
инструментария с целью поддержки автоматической проверки требований на примере 
средства ``Анализатор MIL STD-1553B'' [\ref{opermon_reg}]. Поддержка проверки 
требований для обменов на других каналах связи является перспективой развития 
данной работы.