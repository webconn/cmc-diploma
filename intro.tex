\section{Введение}

\subsection{Актуальность задачи}

\newacronym[longplural={распределённые 
вычислительные системы}]{dcs}{РВС}{распределённая вычислительная система}

В распределённых вычислительных системах (РВС) компоненты взаимодействуют друг 
с другом, используя согласованные и утвержденные разработчиками системы 
протоколы. В этих протоколах содержится информация об абонентах (компонентах 
системы), адресации на каналах, наборе передаваемых сообщений, частотах этих 
сообщений, их форматах, ограничениях на последовательность, описание 
содержимого этих сообщений.

Одна из задач интеграционного тестирования РВС - проверка соблюдения требований 
этих протоколов компонентами системы. Для этого в среде тестирования 
описываются тесты, которые выполняют соотвествующие проверки: для принятых от 
тестируемого компонента сообщений проверяются различные характеристики из 
упомянутых выше. Некоторые детали соблюдения протоколов можно проверить только 
так: передать компоненту сообщение, содержащее параметр N, получить от него в 
ответ сообщение с зависимым параметром М и проверить, что преобразование 
параметра компонентом выполнено правильно. С другой стороны, некоторые 
ограничения (например, частотные характеристики) возможно проверить только при 
регистрации обменов на работающей системе и сравнивании характеристик обменов с 
ограничениями, описанными в протоколах.

Таким образом, целесообразно иметь инструмент, который по регистрируемым обменам 
в реальном времени или по записанной трассе сможет проверить соответствие этих 
обменов требованиям протоколов.

\subsection{Неформальная постановка задачи}

Целью работы является разработка методов и программных средств для анализа 
трасс информационного обмена по МКИО на предмет соответствия зарегистрированных 
обменов и передаваемых в обменах данных требованиям, описанным в протоколах. 
Для этого требуется:

\begin{enumerate}
 \item Провести анализ существующих протоколов информационного 
взаимодействия и предложить набор ограничений/характеристик, которые можно 
проверять в ходе работы анализатора.
 \item Предложить формальное описание требований к обмену и параметру. Отдельно 
учесть возможность автоматической генерации части требований по базе данных 
протокола информационного взаимодействия (БД ПИВ).
 \item Спроектировать инструмент анализа в рамках Анализатора МКИО.
 \item Реализовать результат анализа, провести апробацию.
\end{enumerate}
