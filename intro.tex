\section{Введение}

\subsection{Описание проблемы и актуальность}

Предметом исследования данной работы являются распределённые вычислительные 
системы (РВС) управления сложными техническими объектами (кораблями, 
самолётами, производственными линиями и т.п.). \textit{Распределённая 
вычислительная система} - это набор устройств (вычислительных машин, датчиков и 
актуаторов), объединённых с помощью каналов передачи данных (одного или 
нескольких). 

При разработке и отладке РВС возникает задача проверки корректности 
функционирования системы. Для решения этой задачи используются следующие 
подходы:

\begin{itemize}
 \item \textit{активный} - подразумевающий вмешательство в работу системы. 
Обычно для этого в каналы связи отправляются управляющие сообщения, 
после чего анализируется отклик системы. Этот подход часто используется для 
проверки отдельных узлов, при этом управляющие сообщения моделируют 
процесс функционирования целой системы;
 \item \textit{пассивный} - без вмешательства в работу системы. Проводится 
запись \textit{трассы} обменов - последовательность сообщений и пауз, 
наблюдаемых в каналах связи функционирующей системы, после чего проводится 
анализ трассы на предмет некорректной работы узлов.
\end{itemize}

В обоих подходах после сбора трасс обменов требуется провести их анализ. Это 
можно сделать следующими способами:

\begin{itemize}
 \item сравнением записанной трассы с эталоном;
 \item проверкой соблюдения трассой набора требований.
\end{itemize}

Решение задачи сравнения трассы обменов с эталоном предложено, к примеру, в 
работе [\ref{gordeev_diploma}].

Как правило, описание требований, предъявляемых к РВС, представлено в 
наборе \textit{протоколов}. \textit{Протокол} - документ, содержащий 
информацию о компонентах системы, адресации на каналах передачи данных, наборе 
передаваемых сообщений, их свойствах (форматах, частотах, ограничениях на 
последовательность и т.п.), а также описание содержимого этих сообщений. 
Протоколы согласуются и утверждаются разработчиками РВС. Стоит отметить, 
что часто протоколы представляют собой неформальное текстовое описание, 
непригодное для проведения автоматической проверки. Таким образом, возникает 
необходимость в \textit{формализации} описания требований.

Поскольку в состав РВС может входить большое количество устройств (десятки или 
сотни), количество проверяемых требований может оказаться таким, что анализ 
системы на соответствие им без автоматического средства окажется очень сложным 
или невозможным, что подтверждает актуальность задачи \textit{автоматического} 
анализа системы.

Таким образом, при разработке и отладке РВС полезно иметь инструмент для 
проведения автоматического анализа системы на соответствие требованиям, 
предъявляемым в протоколах.

\subsection{Анализатор МКИО}

В ЛВК разрабатывается инструмент для работы с каналами связи РВС -  
Анализатор МКИО [\ref{opermon_reg}]. Инструмент поддерживает работу с  
различными типами каналов связи, в том числе MIL STD-1553B / МКИО ГОСТ Р 
52070-2003 [\ref{gost_r_52070-2003}], Fibre Channel, CAN и ARINC 429. 
Анализатор МКИО реализован на языке C++ с использованием инструментария Qt 4 
[\ref{qt_ref}].

Среди возможностей инструмента следует отметить следующие:

\begin{itemize}
 \item регистрация обменов в реальном времени;
 \item запись трасс обменов;
 \item получение значений отдельных параметров, передаваемых в сообщениях;
 \item построение графиков значений параметров.
\end{itemize}

Здесь \textit{трасса} - последовательность зарегистрированных сообщений и пауз; 
\textit{параметр} - логическая единица передаваемой информации, которой 
соответствует набор атомарных значений из полезной нагрузки сообщений. Примеры 
параметров для бортовой РВС: скорость, высота над уровнем моря, координаты GPS.

Правила для вычисления значений параметров предлагаются пользователем 
Анализатора МКИО.

Анализатор МКИО используется при разработке и отладке бортовых РВС. Однако, 
актуальная версия Анализатора МКИО не поддерживает автоматическую проверку 
требований протоколов. 

В рамках данной работы будет предложена реализация средства автоматической 
проверки требований протоколов на основе инструмента Анализатор МКИО.