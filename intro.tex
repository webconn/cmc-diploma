\section{Введение}

\subsection{Актуальность задачи}

В РВС компоненты взаимодействуют друг с другом, используя согласованные и 
утвержденные разработчиками системы протоколы. В этих протоколах содержится 
информация об абонентах (компонентах системы), адресации на каналах, наборе 
передаваемых сообщений, частотах этих сообщений, их форматах, ограничениях на 
последовательность, описание содержимого этих сообщений.

Одна из задач интеграционного тестирования РВС - проверка соблюдения этих 
протоколов компонентами системы. Для этого в среде тестирования описываются 
тесты, которые выполняют соотвествующие проверки: для принятых от тестируемого 
компонента сообщений проверяются различные характеристики из упомянутых выше. 
Некоторые детали соблюдения протоколов можно проверить только так: передать 
компоненту сообщение, содержащее параметр N, получить от него в ответ сообщение 
с зависимым параметром М и проверить, что преобразование параметра компонентом 
выполнено правильно. С другой стороны, некоторые ограничения (например, 
частотные характеристики) возможно проверить только при регистрации обменов на 
работающей системе и сравниваи характеристик обменов с ограничениями, описанными 
в протоколах.

Таким образом, целесообразно иметь инструмент, который по регистрируемым обменам 
в реальном времени или по записанной трассе сможет проверить соответствие этих 
обменов требованиям протоколов.
