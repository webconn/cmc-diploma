\subsection{Формат описания входных данных}

Формат входных данных обратно совместим с форматом, использованным в некоторых 
версиях Opermon для описания сообщений и битовых полей на основе спецификации 
ПИВ.

Входные данные представляют собой XML-документ следующего содержания: 
% TODO: схема

\lstinputlisting[caption=Структура файла описания входных 
данных]{input_example.xml}

Атрибуты описания абонента (тег abonent):

\begin{itemize}
 \item identifier - индентификатор абонента (строка - идентификатор Си);
 \item mil1553\_addr - адрес абонента на шине MIL STD-1553B.
\end{itemize}

Атрибуты описания сигнала (тег signal):

\begin{itemize}
 \item identifier - идентификатор сигнала (строка - идентификатор Си);
 \item type - тип данных сигнала (например, int, unsigned int, double); 
приведён для справки;
 \item signed - является ли сигнал знаковым (true|false, по умолчанию true);
 \item twosComplement - записывается ли отрицательное значение в 
дополнительном коде (true|false, по умолчанию false для совместимости со 
старым форматом ПИВ).
\end{itemize}

Тег restrict также может содержать элементы внутри, если это требуется для 
определённого типа ограничений.

Атрибуты сообщения MIL STD-1553B для контроллера (тег mil1553\_contrMessage):

\begin{itemize}
 \item identifier - идентификатор сигнала (строка - индентификатор Си);
 \item direction - направление (input | output - к/от контроллера);
 \item addr  - адрес ОУ (число от 1 до 31);
 \item subaddr - подадрес ОУ (целое число от 1 до 30);
 \item numWords - число слов в сообщении (от 1 до 32).
\end{itemize}

Атрибуты сообщения MIL STD-1553B для оконечного устройства (тег 
mil1553\_termMessage):

\begin{itemize}
 \item identifier - идентификатор сигнала (строка - индентификатор Си);
 \item direction - направление (input | output - к/от контроллера);
 \item subaddr - подадрес ОУ (целое число от 1 до 30);
 \item numWords - число слов в сообщении (от 1 до 32).
\end{itemize}

Стоит заметить, что в описании сообщений MIL STD-1553B для оконечных устройств 
не указан адрес ОУ-получателя сообщения. Это связано с особенностями 
внутреннего устройства используемых БД ПИВ. Для формирования полного заголовка 
сообщения требуется найти два ``полусообщения'' - сообщения 
mil1553\_termMessage у двух абонентов, где атрибуты identifier для сообщений 
совпадают.

Атрибуты описания ограничений для сигнала (тег restrict внутри тега bitfield):

\begin{itemize}
 \item type - тип ограничения (см. Ограничения для сигналов); %TODO: ссылка
 \item value - значение для ограничения (необязательный параметр), зависит от 
типа ограничения;
 \item level - уровень критичности ограничения (info, notice, warning, error).
\end{itemize}


\subsection{Формирование проверяющих объектов}

