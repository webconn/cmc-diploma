\section{Заключение}

В рамках настоящей работы было сделано следующее:

\begin{itemize}
 \item проведён анализ существующих протоколов информационного взаимодействия и 
предложен набор требований, предъявляемых к обменам и параметрам;
 \item предложено формальное описание требований к обменам и параметрам;
 \item разработан метод проверки требований;
 \item выполнена программная реализация анализатора требований и 
проведена апробация разработанного инструмента на сгенерированных 
трассах обменов, а также показана возможность использования инструмента при 
анализе систем в режиме реального времени.
\end{itemize}

%TODO формулировка
В качестве перспектив развития данной работы можно предложить следующие:

\begin{enumerate}
 \item Усовершенствование интеграции с  инструментами Анализатора MIL STD-1553B:
 \begin{itemize}
  \item установка связей между строками отчёта анализатора и списком обменов 
для выделения в пользовательском интерфейсе обменов с обнаруженными проблемами;
  \item разработка пользовательского интерфейса для редактирования требований к 
обменам, минуя подготовку файла протокола.
 \end{itemize}
 
 \item Усовершенствование пользовательского интерфейса вкладки с отчётом 
анализатора:
 \begin{itemize}
  \item добавление дополнительных колонок данных: ``Тип ограничения'', 
``Канал'' и т.п.;
  \item возможность сортировки по колонкам в таблице с отчётом анализатора;
  \item возможность фильтровать элементы отчёта по содержанию.
 \end{itemize}

 \item Расширение сферы применения разработанного средства на другие каналы 
информационного обмена.
\end{enumerate}

