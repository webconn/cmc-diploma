\section{Примеры описаний ограничений}
\label{sec:listings}

\lstinputlisting[label={input_example},caption={Структура файла описания 
входных данных}]{input_example.xml}

\lstinputlisting[label={msg_frequency},caption={Частота появления 
сообщения}]{config_examples/msg_frequency.xml}

\lstinputlisting[label={msg_errors}, caption={Ошибочные состояния 
сообщения}]{config_examples/msg_errors.xml}

\lstinputlisting[label={msg_sequence}, caption={Последовательность 
сообщений}]{config_examples/msg_sequence.xml}

\lstinputlisting[label={msg_crc}, caption={Проверка контрольной 
суммы}]{config_examples/msg_crc.xml}

\lstinputlisting[label={param_frequency}, caption=Частота появления 
сообщения]{config_examples/param_frequency.xml}

\lstinputlisting[label={param_thresholds}, caption=Пороговые значения 
параметра]{config_examples/param_thresholds.xml}

\lstinputlisting[label={param_const}, caption=Равенство значения параметра 
константе]{config_examples/param_const.xml}

\lstinputlisting[label={param_errorvalue}, caption=Ошибочное значение 
параметра]{config_examples/param_errorvalue.xml}

\lstinputlisting[label={param_smooth}, caption=Гладкость значения 
параметра]{config_examples/param_smooth.xml}

\lstinputlisting[label={param_groups}, caption=Связанные 
параметры]{config_examples/param_groups.xml}

\lstinputlisting[label={param_autoinc}, caption=Автоинкремент]{
config_examples/param_autoinc.xml }