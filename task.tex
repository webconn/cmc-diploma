\section{Постановка задачи}

Целью работы является проектирование и разработка программных средств для 
анализа трасс информационного обмена по каналу MIL STD-1553B на предмет 
соответствия зарегистрированных обменов и передаваемых в обменах данных 
требованиям, описанным в протоколах. Для этого требуется:

\begin{enumerate}
 \item Провести анализ существующих протоколов информационного 
взаимодействия и предложить набор ограничений/характеристик, которые можно 
проверять в ходе работы анализатора.
 \item Предложить формальное описание требований к обмену и параметру. Отдельно 
учесть возможность автоматической генерации части требований по базе данных 
протокола информационного взаимодействия (БД ПИВ).
 \item Спроектировать средство проведения анализа в рамках инструмента 
Анализатор MIL STD-1553B.
 \item Реализовать инструмент анализа, провести апробацию.
\end{enumerate}

К инструменту анализа предъявляется следующий набор требований, связанных с 
возможными способами прикладного применения инструмента:

\begin{itemize}
 \item описание конфигурации анализатора должно быть простым для 
понимания, создания и редактирования как вручную пользователем анализатора, так 
и с применением программного обеспечения для автоматического составления 
описания протоколов;
 \item формат описания конфигурации анализатора должен быть совместимым с 
используемым на текущий момент форматом описания протокола для Анализатора 
MIL STD-1553B для возможности переиспользования существующего программного 
обеспечения автоматического составления описания протоколов;
 \item анализатор должен иметь возможность функционирования в режиме 
регистрации обмена в реальном времени при условии выполнении анализа на 
рабочей станции пользователя Анализатора MIL STD-1553B 
и проверки достаточно большого количества требований. (Это значит, что на 
современном персональном компьютере при регистрации обмена в реальном времени 
анализ должен выполняться без заметных задержек с точки зрения пользователя);
 \item анализатор должен быть готовым к возможным расширениям 
 функционала (определению новых типов требований к обменам и параметрам). При 
добавлении новых типов ограничений пользователь должен иметь возможность 
использовать описания требований, подготовленные для предыдущей версии  
анализатора (возможно, с незначительными изменениями).
\end{itemize}