\section{Введение}

\subsection{Неформальная постановка задачи}

Целью работы является разработка методов и программных средств для 
анализа трасс данных обменов по МКИО с учётом заранее определённых требований. 
Примеры возможных требований:

\begin{itemize}
 \item асинхронность сообщения, либо частота появления сообщения;
 \item соответствие контрольных сумм;
 \item гладкость передаваемого значения (ограничение на скорость изменения 
значения);
 \item автоматичеcкое увеличение (автоинкремент) передаваемого значения во 
времени
\end{itemize}

Трасса состоит из слов (?) с признаками КС, ОС или СД. Из последовательности 
слов согласно одному из 10 шаблонов, приведённых в ГОСТ Р 52070-2003, 
составляются обмены, для которых выделяется набор полей:

\begin{itemize}
 \item время начала обмена;
 \item продолжительность обмена;
 \item формат обмена - номер шаблона из ГОСТ Р 52070-2003
 \item адрес и подадрес абонента-отправителя и абонента-получателя (отдельно 
определяется фиктивный адрес КШ, так как он может не иметь реального адреса на 
шине);
 \item количество слов данных, переданных в рамках обмена;
 \item признак и тип ошибки, произошедшей при выполнении обмена;
 \item переданные данные.
\end{itemize}

Анализатору сообщается набор требований к конкретным сообщениям. 
Допускается также, что МКИО может функционировать в нескольких различных 
режимах, при этом для каждого режима формируется свой набор требований и 
условия переходов из одного режима в другой.

\subsection{Определения и обозначения}

% Определения взяты из дипломной работы Гордеева (2008) с некоторыми 
% изменениями и добавлениями

\textbf{Обмен} - ($M = <t, d, f, a_{src}, a_{dst}, s_{src}, s_{dst}, sz, e, p 
>$) - атомарный завершённый обмен информацией между абонентами канала МКИО, в 
соответствии с протоколом обмена по ГОСТ Р 52070-2003 
[\ref{gost_r_52070-2003}]. Понятие обмена соответствует термину 
\textit{сообщение} ГОСТ. С точки зрения анализа последовательностей обменов, 
каждый обмен характеризуется следующими атрибутами:

\begin{itemize}
 \item $t$ - время начала обмена (мкс);
 \item $d$ - продолжительность обмена (мкс);
 \item $f$ - формат (тип) обмена по ГОСТ Р 52070-2003 - целое число в интервале 
$[1, 10]$;
 \item $a_{src}, a_{dst}$ - адреса абонента-отправителя и абонента-получателя 
данных в рамках обмена - целые числа на отрезке $[0, 32]$ ($[0, 31]$ - 
допустимое множество адресов для МКИО, $32$ - фиктивный адрес КШ (контроллера 
шины);
 \item $s_{src}, s_{dst}$ - подадреса абонента-отправителя и 
абонента-получателя данных в рамках обмена - целые числа на отрезке $[1, 31]$;
 \item $sz$ - количество слов данных, передаваемых в сообщении - целое число на 
отрезке $[0, 31]$;
 \item $e$ - признаки обмена - подмножество множества $E$, 
соответствующего набору признаков, передаваемых в ОС:
 \begin{itemize}
  \item $Err$ - ``Ошибка в сообщении'' - признак достоверности принятых данных;
  \item $Mntn$ - ``Запрос на обслуживание'' - требование ОУ или связанного с 
ним абонента (абонентов) на обслуживание;
  \item $Grp$ - ``Принята групповая команда'' - указывает на приём ОУ 
достоверной групповой команды;
  \item $Busy$ - ``Абонент занят'' - состояние занятости ОУ или интерфейса 
абонента, которое может повлиять на обмен информацией по каналу;
  \item $Aflt$ - ``Неисправность абонента'' - техническое состояние абонента, 
связанного с ОУ;
  \item $Eflt$ - ``Неисправность ОУ'' - техническое состояние ОУ;
 \end{itemize}
 \item $p$ - полезная нагрузка (переданные данные в СД) - последовательность 
слов длины $M_sz$.
\end{itemize}

Атрибуты сообщения $M$ обозначаются так: $M_t$, $M_d$, ..., $M_{a_{src}}$, ..., 
$M_p$. Дополнительно используются обозначения границ интервалов времени 
$M_{start} = M_t$, $M_{end} = M_{start} + M_d$ и обозначение продолжительности 
обмена $|M| = M_d$. Если в названии сообщения используются индексы (например, 
$M^i$), то атрибуты обозначаются так: $M^i_t$, $M^i_d$, ..., 
$M^i_{a_{src}}$, $M^i_p$.


% TODO: обсудить момент форматизации контрольной суммы, где она появляется и 
% т.д.

Дополнительно в сообщении $M$ может передаваться его контрольная сумма, 
значение которой будем в дальнейшем обозначать $M_{ch}$.

\textbf{Экземпляр обмена} - совокупность $M$ значений всех атрибутов обмена. 
Экземпляр обмена характеризуется типом $Type(M)$, временем начала $M_t$, 
продолжительностью $M_d$, признаками $M_e$ и полезной нагрузкой $M_p$.

\textbf{Тип обмена} ($T = Type(M)$) - совокупность значений атрибута обмена 
$<f, a_{src}, a_{dst}, s_{src}, s_{dst}, sz>$. Каждый экземпляр обмена, таким 
образом, соответствует одному и только одному типу данных.

\textbf{Трасса обменов} $T = \{  M^1, M^2, ..., M^n \} : M^i_{end} \leq 
M^{i+1}_{start} \forall i = \overline{1,n-1}$ - конечная последовательность 
обменов. Время начала обменов $M^i_t$ отсчитывается относительно момента начала 
записи трассы обмена.

% TODO: Надо как-то назвать задаваемую последовательность обменов...

\textbf{Описание цепочки обменов} $D$ - конечный автомат, множеством состояний 
и входным алфавитом которого является подмножество множества типов обмена. 
Определяя это множество, а также начальное состояние, множество конечных 
состояний и функцию перехода, пользователь определяет логически связанные 
последовательности обменов в трассе.

\textbf{Цепочка обменов} $C$ - логически связанная последовательность обменов 
в трассе (не обязательно непосредственно следующих друг за другом), 
определяемая описанием цепочки $D$.

В дальнейшем будем пользоваться следующим обозначением: описанию цепочки $D_i$ 
в трассе $T$ соответствует последовательность цепочек $C_i = \{ C_i^1, ..., 
C_i^k \}$.

% TODO: Здесь тоже корявое название

\textbf{Функция корректности последовательности цепочек} $R$ - функция $R: 
{C^1, ..., C^q} \rightarrow \{ 1, 0 \}$, определяющее корректность входной 
последовательности цепочек, т. е. $R(C) = 1 \Leftrightarrow C$ - корректная 
последовательность цепочек. Требования корректности определяются пользователем 
для проверки непосредственно передаваемых данных, в том числе с учётом 
типов, порядка и времени их передачи.

\subsection{Формальная постановка задачи}

\textbf{Исходные данные}: зарегистрированная трасса обменов на канале МКИО $T = 
\{ M^1, ..., M^n \}$, множество описаний цепочек $D = \{ D_1, ..., D_m \}$ и 
функция корректности последовательностей цепочек $R$.

\textbf{Требуется}: разработать набор алгоритмов и реализовать программные 
средства, обеспечивающие решение следующих задач:

\begin{itemize}
 \item поиск цепочек $C_i$ в трассе $T$ согласно 
описаниям цепочек $D_i$;
 \item вычисление статистических характеристик цепочек обменов $C_i$:
 \begin{itemize}
  \item частота появления цепочек $f_i$;
 \end{itemize}
 \item проверка корректности последовательностей цепочек $C_i$ с помощью 
функции $R$.

\end{itemize}

\subsection{Уточнение исходных данных}

\subsubsection{Режимы работы системы}

Множество описаний цепочек делится дополнительно на подмножества, образующие 
\textit{режимы} работы анализируемой системы. Дополнительно к каждому описанию 
цепочки дописывается поле, содержащее новое значение режима работы (в том 
случае, если появление данной цепочки это подразумевает).

\subsubsection{Функция корректности последовательности цепочки}

Для упрощения задачи полезно уточнить определение функции корректности 
последовательности так, чтобы она требовала меньше входных данных с сохранением 
всех свойств.

Введём дополнительно определение \textbf{подцепочки} $\hat{C}^t_k$ цепочки $C = 
\{M^1, ..., M^n\}$ длительности $t$ как максимальной подпоследовательности 
последовательности $\{M^j\}^{j=k}_{k+p}$, где время между началом первого и 
последнего сообщения не превышает $t$: $M^{k+p}_{start} - M^k_{start} \leq t$. 
Требование максимальности необходимо для попадания в подцепочку всех обменов на 
отрезке времени $[M^k_{start}, M^k_{start}+ t]$.

Считается, что функция проверки корректности $R$ может быть одной из 
нижеперечисленных, определённых на соответствующих множествах обменов, 
имеющих требуемые в формулировках атрибуты:

\begin{itemize}
 \item $R_p(M^i, M^{i+1}, f_{Type(M)})$ - функция, проверяющая асинхронность 
сообщения, либо соответствие реальной частоты появления обмена $M$ заранее 
заданной $f_{Type(M)}$: $R_p(M^i, M^{i+1}, f_{Type(M)}) = 1 \Leftrightarrow 
(f_{Type(M)} = 0) \vee (M^{i+1}_{start} - M^i_{start} \geq 
{f_{Type(M)}}^{-1}, f_{Type(M)} > 0)$. $R_p$ определена на множестве всех 
обменов;

 \item $R_c(M^i)$ - функция, устанавливающая соответствие подсчитанной 
контрольной суммы тела обмена и переданной в обмене: $R_c(M) = 1 
\Leftrightarrow checksum(M^i) = M^i_{ch}$. $R_c$ определена на множестве 
обменов, имеющих атрибут $M_{ch}$;

% TODO: с этим товарищем всё не так гладко, нужно определить длину 
% или длительность последовательности обменов одного типа
 \item $R_{ai}(\hat{C}^i_{t_{Type(M)}})$ - функция, определяющая 
автоматичеcкое увеличение (автоинкремент) передаваемого значения во 
времени не реже, чем каждый интервал времени длительности $t_{Type(M)}$: 
$R_{ai}(\hat{C}^i_{t_{Type(M)}}) = 1 \Leftrightarrow (M^{i+p}_{start} - 
M^i_{start} > t_{Type(M)}) \wedge (M^{i+p}_{ai} - M^i_{ai} > 0)$, где $p$ - 
длина подцепочки $\hat{C}^i$, $M^j_v$ - заранее определённый увеличиваемый 
атрибут обмена. $R_{ai}$ определена на множестве обменов, имеющих атрибут 
$M_{ai}$.
\end{itemize}

Константы $f_{Type(M)}$ и $t_{Type(M)}$ можно занести внутрь функций $R$. Также 
стоит расширить функции $R_c$ и $R_{ai}$ на множество всех возможных обменов, 
доопределив их значением 1. Таким образом, мы получаем функции 
$R^{Type(M)}_p(M^{i}, M^{i+1})$, $R^{Type(M)}_c(M)$, $R^{Type(M)}_{ai}(M)$

Наличие атрибутов $M_{ch}$ и $M_{ai}$ определяется типом сообщения $M$.

Таким образом, в общем случае функция обмена существенно зависит от:

\begin{itemize}
 \item типа обрабатываемого обмена $Type(M)$;
 \item одного обмена, или двух обменов, или подцепочки обменов длительности $t$,
\end{itemize}

что упрощает описание множества определения функции $R$.

% далее следует немного треша aka выписки и мысли вслух
\newpage

\iffalse

\begin{enumerate}
 \item КС - СД - СД - ... - СД - [t] - ОС - формат 1
 \item КС - [t] - ОС - СД - СД - ... - СД - формат 2
 \item КС - КС - [t] - ОС - СД - ... - СД - [t] - ОС - формат 3
 \item КС - [t] - ОС - формат 4
 \item КС - [t] - ОС - СД - формат 5
 \item КС - СД - [t] - ОС - формат 6
 \item КС - СД - ... - СД - формат 7 (групповой здесь и далее)
 \item КС - КС - [t] - ОС - СД - ... - СД - формат 8
 \item КС - формат 9
 \item КС - СД - формат 10
\end{enumerate}

Каждая последовательность может содержать не более 32 слов данных.

Каждая из вышепредложенных последовательностей соответствует своему формату сообщения:

\begin{enumerate}
 \item Формат 1 - передача данных от контроллера шины (КШ) к оконечному устройству (ОУ)
 \item Формат 2 - передача данных от ОУ к КШ
 \item Формат 3 - передача данных от ОУ к ОУ
 \item Формат 4 - передача команды управления (КУ)
 \item Формат 5 - передача КУ и приём СД от ОУ
 \item Формат 6 - передача КУ с СД оконечному устройству
 \item Формат 7 - передача данных от КШ к оконечным устройствам
 \item Формат 8 - передача данных от ОУ оконечным устройствам
 \item Формат 9 - передача КУ группе ОУ
 \item Формат 10 - передача КУ с СД группе ОУ
\end{enumerate}

Считаем, что КШ на линии один и не меняется.
\fi