\section{Анализ предметной области}

Как правило, в таких системах компоненты взаимодействуют друг с другом в 
соответствии с набором протоколов. \textit{Протокол} - документ, содержащий 
информацию о компонентах системы, адресации на каналах передачи данных, наборе 
передаваемых сообщений, их свойствах (форматах, частотах, ограничениях на 
последовательность и т.п.), а также описание содержимого этих сообщений. 
Протоколы согласуются и утверждаются разработчиками РВС.

Одной из задач интеграционного тестирования РВС является проверка соблюдения 
компонентами требований, описанных в этих протоколах. Стоит отметить, 
что часто протоколы представляют собой неформальное текстовое описание, 
непригодное для проведения автоматической проверки. Тем не менее, из 
используемых на практике протоколов можно выделить общий набор типов 
требований, 
предъявляемых к сообщениям и передаваемым в них данным. После предложения 
формального представления этих требований появляется возможность проведения 
автоматического анализа системы на их соблюдение.

Несоблюдение системой требований из протоколов может означать, что при 
проектировании или разработке РВС была допущена ошибка, или какое-либо 
устройство (или несколько устройств) в системе работают некорректно.